\documentclass[oneside, 12pt, a4paper]{article}

% 导入必要的宏包
\usepackage{ctex} % 用于提供中文支持
\usepackage{geometry} % 用于设置页面版式
\usepackage{hyperref} % 用于创建超链接

% 设置页面边距
\geometry{
    a4paper, 
    left=3cm, 
    right=3cm, 
    top=3cm, 
    bottom=3cm
}

% 设置超链接样式
\hypersetup{
    colorlinks=true,  % 将超链接文本设置为彩色
    linkcolor=blue,   % 将超链接文本的颜色设置为蓝色
    citecolor=blue,   % 将引用文本的颜色设置为蓝色
    urlcolor=blue,    % 将超链接地址的颜色设置为蓝色
    pdfauthor={凌杰}, % 设置 PDF 文档的作者信息
    pdftitle={\LaTeX 排版示例} % 设置 PDF 文档的标题信息
}
% 设置文档元信息
\title{\LaTeX 排版示例}
\author{凌杰}
\date{\today}
% 开始组织文档内容
\begin{document}
    \section{正文样式示例}
        \subsection{正文段落}
            这是一段正文,它没有特殊的样式,我们只需\textbf{直接写在章节标题下面}即可。在这里继续添加文本,丰富一些内容以展示段落效果。

            这是另一段正文,它也没有特殊的样式,它\textit{与上一段正文之间用空行分割}即可。在这里\underline{继续添加文本,丰富一些内容以展示段落效果}。

            我们还可以在正文中插入一个能跳转到 \href{https://www.tug.org/texlive/}{TeX Live 官方网站}的超链接。
        \subsection{引言文本}       
            下面将显示的是一段引言文本的示例:
            \begin{quote}
                \textit{“\LaTeX 是一种排版系统,它使用TeX排版引擎,并遵循一套文档类和宏包的规范,以实现文档的自动化排版。”}
        
                {\raggedleft -- 高德纳(Donald E. Knuth) \par}
            \end{quote}

            下面是一个较长的引用文本示例:
            \begin{quotation}
                \textit{“Leslie Lamport 在 1980 年代初期开发了 \LaTeX 系统,以简化 \TeX 的使用。\LaTeX 提供了一套宏和命令,使用户能够专注于文档内容,而无需过多关注排版细节。\LaTeX 迅速成为学术界和技术文档撰写的标准工具,尤其在数学、计算机科学等领域得到了广泛应用。}

                \textit{\LaTeX 的设计理念强调结构化文档和内容优先,这使得它在处理复杂文档时表现出色。通过使用各种宏包,用户可以轻松地扩展 \LaTeX 的功能,以满足不同的排版需求。今天,\LaTeX 已经发展成为一个庞大的生态系统,拥有丰富的资源和活跃的社区支持。”}
                
                {\raggedleft -- 节选自维基百科 \par}
            \end{quotation}
        \subsection{列表项目}
        \subsection{注释说明}
    \section{图表元素示例}
        \subsection{插图元素}
        \subsection{表格元素}
    \section{专业内容示例}
        \subsection{定理证明}
        \subsection{数学公式}
        \subsection{算法描述}
        \subsection{代码高亮}
\end{document}
