\documentclass[oneside, 12pt, a4paper]{article}

\usepackage[indentfirst=false]{ctex} % 用于提供中文支持
\usepackage{geometry} % 用于设置页面版式
\usepackage{hyperref} % 用于创建超链接

\hypersetup{
    colorlinks=true,
    linkcolor=blue,
    citecolor=blue,
    urlcolor=blue,
    pdfauthor={凌杰},
    pdftitle={\LaTeX 排版示例},
}
% 设置页面边距
\geometry{a4paper, left=3cm, right=3cm, top=3cm, bottom=3cm}
% 设置文档元信息
\title{\LaTeX 排版示例}
\author{凌杰}
\date{\today}
% 开始组织文档内容
\begin{document}
    \section{正文样式示例}
        \subsection{正文段落}
        这是一段正文,它没有特殊的样式,我们只需\textbf{直接写在章节标题下面}即可。在这里继续添加文本,丰富一些内容以展示段落效果。

        这是另一段正文,它也没有特殊的样式,它\textit{与上一段正文之间用空行分割}即可。在这里\underline{继续添加文本,丰富一些内容以展示段落效果}。

        我们还可以在正文中插入一个能跳转到 \href{https://www.tug.org/texlive/}{TeX Live 官方网站}的超链接。
        \subsection{引言文本}
        \subsection{列表项目}
        \subsection{注释说明}
    \section{图表元素示例}
        \subsection{插图元素}
        \subsection{表格元素}
    \section{专业内容示例}
        \subsection{定理证明}
        \subsection{数学公式}
        \subsection{算法描述}
        \subsection{代码高亮}
\end{document}
