% 设置文档类为 article,指定单面打印、12pt 字体大小和 A4 纸张大小
\documentclass[oneside, 12pt, a4paper]{article}

% 导入必要的宏包
\usepackage{ctex}     % 用于提供中文支持
\usepackage{geometry} % 用于设置页面版式
\usepackage{hyperref} % 用于创建超链接
\usepackage{amsmath}  % 提供数学公式支持
\usepackage{graphicx} % 用于插入图片
\usepackage{wrapfig}  % 用于实现文字环绕
\usepackage{multirow} % 用于表格中的单元格合并

% 设置页面边距
\geometry{
    left=3cm,      % 设置左边距
    right=3cm,     % 设置右边距
    top=3cm,       % 设置上边距
    bottom=3cm     % 设置下边距
}

% 设置超链接样式
\hypersetup{
    colorlinks=true,             % 将超链接文本设置为彩色
    linkcolor=blue,              % 将超链接文本的颜色设置为蓝色
    citecolor=blue,              % 将引用文本的颜色设置为蓝色
    urlcolor=blue,               % 将超链接地址的颜色设置为蓝色
    pdfauthor={凌杰},            % 设置 PDF 文档的作者信息
    pdftitle={\LaTeX 排版示例}   % 设置 PDF 文档的标题信息
}
\marginparwidth=2cm       % 设置边注的宽度
\marginparsep=0.2cm       % 设置边注与正文的间距
\setcounter{tocdepth}{2}  % 设置目录的层级深度

% 设置文档元信息
\title{\LaTeX 排版示例}
\author{凌杰}
\date{\today}
% 开始组织文档内容
\begin{document}
    \maketitle       % 生成标题
    \tableofcontents % 生成目录
    \newpage         % 插入分页符
    \section{正文样式示例}
        \subsection{正文段落}
            \noindent 这是一段正文,它没有特殊的样式,我们只需\textbf{直接写在章节标题下面}即可。在这里继续添加文本,丰富一些内容以展示段落效果。

            这是另一段正文,它也没有特殊的样式,它\textit{与上一段正文之间用空行分割}即可。在这里\underline{继续添加文本,丰富一些内容以展示段落效果}。

            我们还可以在正文中插入一个能跳转到 \href{https://www.tug.org/texlive/}{TeX Live 官方网站}的超链接。
        \subsection{引言文本}       
            \noindent 下面将显示的是一段引言文本的示例:
            \begin{quote}
                \textit{“\LaTeX 是一种排版系统,它使用TeX排版引擎,并遵循一套文档类和宏包的规范,以实现文档的自动化排版。”}
        
                {\raggedleft -- 高德纳(Donald E. Knuth) \par}
            \end{quote}

            \noindent 下面是一个较长的引用文本示例:
            \begin{quotation}
                \textit{“Leslie Lamport 在 1980 年代初期开发了 \LaTeX 系统,以简化 \TeX 的使用。\LaTeX 提供了一套宏和命令,使用户能够专注于文档内容,而无需过多关注排版细节。\LaTeX 迅速成为学术界和技术文档撰写的标准工具,尤其在数学、计算机科学等领域得到了广泛应用。}

                \textit{\LaTeX 的设计理念强调结构化文档和内容优先,这使得它在处理复杂文档时表现出色。通过使用各种宏包,用户可以轻松地扩展 \LaTeX 的功能,以满足不同的排版需求。今天,\LaTeX 已经发展成为一个庞大的生态系统,拥有丰富的资源和活跃的社区支持。”}
                
                {\raggedleft -- 节选自维基百科 \par}
            \end{quotation}
        \subsection{列表项目}
            \noindent 下面是一个有序列表的示例:
            \begin{enumerate}
                \item 第一个选项
                \item 第二个选项
                \item 第三个选项
            \end{enumerate}

            \noindent 下面是一个无序列表的示例:
            \begin{itemize}
                \item 第一个选项
                \item 第二个选项
                \item 第三个选项
            \end{itemize}

            \noindent 下面是一个描述列表的示例:
            \begin{description}
                \item[第一个选项:] 这是第一个选项的描述内容
                \item[第二个选项:] 这是第二个选项的描述内容
                \item[第三个选项:] 这是第三个选项的描述内容
            \end{description}

            \noindent 下面是一个多层次列表的示例:
            \begin{enumerate}
                \item 第一个选项
                \item 第二个选项
                \item \begin{enumerate}
                    \item 第一个选项
                    \item 第二个选项
                    \item \begin{enumerate}
                        \item 第一个选项
                        \item 第二个选项
                        \item 第三个选项
                    \end{enumerate}
                \end{enumerate}
            \end{enumerate}
        \subsection{注释说明}
            \noindent 这是一段要被脚注的文本。
            \footnote{这是脚注的内容。}

            \noindent 这是一段要被边注的文本。
            \marginpar{这是右侧边注内容。}
        \newpage         % 插入分页符
        
    \section{图表元素示例}
        \subsection{插图元素}
            \begin{wrapfigure}{r}{0.4\textwidth}
                \vspace{-14pt} % 调整图片与上方文本的间距
                \centering
                \includegraphics[width=\linewidth]{./img/img.png}
                \caption{勾股定理示意图}
                \label{fig:勾股定理示意图}
            \end{wrapfigure}
            \noindent
            勾股定理,是一个基本的几何定理,指直角三角形的两条直角边的平方和等于斜边的平方。中国古代称直角三角形为勾股形,并且直角边中较小者为勾,另一长直角边为股,斜边为弦,所以称这个定理为勾股定理,也有人称商高定理,其几何关系如图\ref{fig:勾股定理示意图}所示。

            在这里,我们使用了 wrapfig 宏包提供的 wrapfigure 环境,该环境会自动将图片元素放置在右侧,并让文字环绕在图片的左侧。
        \newpage         % 插入分页符
        \subsection{表格元素}
            \noindent
            勾股定理的公式有多种不同的表达形式,下面的表格\ref{tab:pythagorean_formulas}总结了一些常见的勾股定理公式。
            \begin{table}[htbp]
                \centering  % 表格居中
                \caption{勾股定理公式汇总} % 表格标题
                \label{tab:pythagorean_formulas} % 表格标签
                \vspace{0.3cm} % 设置表格行间距
                \begin{tabular}{|p{6cm}|p{8cm}|} % 定义两列,均为左对齐,并设置列宽
                    \hline   % 表格行间线
                    \textbf{公式名称} & \textbf{数学表达式} \\
                    \hline   % 表格行间线
                    标准形式 & \( a^2 + b^2 = c^2 \) \\
                    \hline   % 表格行间线
                    求斜边 \(c\) & \( c = \sqrt{a^2 + b^2} \) \\
                    \hline  % 表格行间线
                    求直角边 \(a\) & \( a = \sqrt{c^2 - b^2} \) \\
                    \hline  % 表格行间线
                    求直角边 \(b\) & \( b = \sqrt{c^2 - a^2} \) \\
                    \hline  % 表格行间线
                    三角函数形式(正弦与余弦) & \( \sin^2(\theta) + \cos^2(\theta) = 1 \) \\
                    \hline  % 表格行间线
                    向量形式(二维空间) & \( \|\mathbf{v}\|^2 = v_x^2 + v_y^2 \) \\
                    \hline  % 表格行间线   
                \end{tabular}
            \end{table}

            \begin{table}[htbp]
                \centering
                \caption{勾股定理公式汇总(单元格合并示例)}
                \label{tab:pythagorean_formulas_merge}
                \vspace{0.3cm}
                \begin{tabular}{|p{6cm}|p{8cm}|}
                    \hline
                    \textbf{公式名称} & \textbf{数学表达式} \\
                    \hline
                    标准形式 & \( a^2 + b^2 = c^2 \) \\
                    \hline                    
                    %------ 这里开始合并三行 ------
                    \multirow{3}{*}{求边公式} 
                        & \( c = \sqrt{a^2 + b^2} \) \\ \cline{2-2}
                        & \( a = \sqrt{c^2 - b^2} \) \\ \cline{2-2}
                        & \( b = \sqrt{c^2 - a^2} \) \\ 
                    \hline
                    %------ 合并结束 ------                    
                    三角函数形式(正弦与余弦) & \( \sin^2(\theta) + \cos^2(\theta) = 1 \) \\ 
                    \hline
                    向量形式(二维空间) & \( \|\mathbf{v}\|^2 = v_x^2 + v_y^2 \) \\
                    \hline
                \end{tabular}
            \end{table}
        \newpage         % 插入分页符
    \section{专业内容示例}
        \subsection{定理证明}
        \subsection{数学公式}
        \subsection{算法描述}
        \subsection{代码高亮}
\end{document}
